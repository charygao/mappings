\documentclass{article}
\usepackage{xcolor}
\usepackage[no-math]{fontspec}
\setmainfont{Noto Serif}
\usepackage[Georgian, Latin, Punctuation, Diacritics]{ucharclasses}
%-------------------
\pagecolor{purple!6}
%===============
\setmainfont{Noto Serif}
\newfontfamily\geofontipa[Mapping=georgian-to-latinipa]{Noto Serif}[Colour=blue]
\newfontfamily\geofontiso[Mapping=georgian-to-latiniso]{Noto Serif}[Colour=red]
\newfontfamily\geofontpkb[Mapping=georgian-to-latinpkb]{Noto Serif}[Colour=brown]
\newfontfamily\geofont{Noto Serif Georgian}
\newfontfamily\geomfont[Mapping=latin-to-georgian]{Noto Serif Georgian}
\newcommand\tgeoi[1]{{\geofontipa #1}}
\newcommand\tgeoii[1]{{\geofontiso #1}}
\newcommand\tgeoiii[1]{{\geofontpkb #1}}
\newcommand\textgeo[1]{{\geofont #1}}
\newcommand\textgeom[1]{{\geomfont #1}}
%-------------------
\setTransitionsFor{Georgian}{\geofont}{ }%space(=glue) for line-wrapping
\setTransitionsForLatin{\normalfont}{}
\setTransitionsForPunctuation{\geofontipa}{\normalfont\normalsize\normalcolor}
\setTransitionsForDiacritics{\normalfont}{}


%===============


%-------------------
\begin{document}
\uccoff

\begin{tabular}{rcccrl}
 & Georgian & IPA & ISO & pkb & \\
1 & \textgeo{ა}  & \tgeoi{ა} & \tgeoii{ა} & a $\to$ \textgeom{a} &  $\to$ \tgeoiii{ა} \\
2 & \textgeo{ბ}  & \tgeoi{ბ} & \tgeoii{ბ} & b $\to$ \textgeom{b} &  $\to$ \tgeoiii{ბ} \\
3 & \textgeo{გ}  & \tgeoi{გ} & \tgeoii{გ} & g $\to$ \textgeom{g} &  $\to$ \tgeoiii{გ} \\
4 & \textgeo{დ}  & \tgeoi{დ} & \tgeoii{დ} & d $\to$ \textgeom{d} &  $\to$ \tgeoiii{დ} \\
5 & \textgeo{ე}  & \tgeoi{ე} & \tgeoii{ე} & e $\to$ \textgeom{e} &  $\to$ \tgeoiii{ე} \\
6 & \textgeo{ვ}  & \tgeoi{ვ} & \tgeoii{ვ} & v $\to$ \textgeom{v} &  $\to$ \tgeoiii{ვ} \\
7 & \textgeo{ზ}  & \tgeoi{ზ} & \tgeoii{ზ} & z $\to$ \textgeom{z} &  $\to$ \tgeoiii{ზ} \\
8 & \textgeo{ჱ}  & \tgeoi{ჱ} & \tgeoii{ჱ} & ee $\to$ \textgeom{ee} &  $\to$ \tgeoiii{ჱ} \\
9 & \textgeo{თ}  & \tgeoi{თ} & \tgeoii{თ} & th $\to$ \textgeom{th} &  $\to$ \tgeoiii{თ} \\
10 & \textgeo{ი}  & \tgeoi{ი} & \tgeoii{ი} & i $\to$ \textgeom{i} &  $\to$ \tgeoiii{ი} \\
11 & \textgeo{კ}  & \tgeoi{კ} & \tgeoii{კ} & k $\to$ \textgeom{k} &  $\to$ \tgeoiii{კ} \\
12 & \textgeo{ლ}  & \tgeoi{ლ} & \tgeoii{ლ} & l $\to$ \textgeom{l} &  $\to$ \tgeoiii{ლ} \\
13 & \textgeo{მ}  & \tgeoi{მ} & \tgeoii{მ} & m $\to$ \textgeom{m} &  $\to$ \tgeoiii{მ} \\
14 & \textgeo{ნ}  & \tgeoi{ნ} & \tgeoii{ნ} & n $\to$ \textgeom{n} &  $\to$ \tgeoiii{ნ} \\
15 & \textgeo{ჲ}  & \tgeoi{ჲ} & \tgeoii{ჲ} & j $\to$ \textgeom{j} &  $\to$ \tgeoiii{ჲ} \\
16 & \textgeo{ო}  & \tgeoi{ო} & \tgeoii{ო} & o $\to$ \textgeom{o} &  $\to$ \tgeoiii{ო} \\
17 & \textgeo{პ}  & \tgeoi{პ} & \tgeoii{პ} & p $\to$ \textgeom{p} &  $\to$ \tgeoiii{პ} \\
18 & \textgeo{ჟ}  & \tgeoi{ჟ} & \tgeoii{ჟ} & zh $\to$ \textgeom{zh} &  $\to$ \tgeoiii{ჟ} \\
19 & \textgeo{რ}  & \tgeoi{რ} & \tgeoii{რ} & r $\to$ \textgeom{r} &  $\to$ \tgeoiii{რ} \\
20 & \textgeo{ს}  & \tgeoi{ს} & \tgeoii{ს} & s $\to$ \textgeom{s} &  $\to$ \tgeoiii{ს} \\
21 & \textgeo{ტ}  & \tgeoi{ტ} & \tgeoii{ტ} & t $\to$ \textgeom{t} &  $\to$ \tgeoiii{ტ} \\
22 & \textgeo{ჳ}  & \tgeoi{ჳ} & \tgeoii{ჳ} & w $\to$ \textgeom{w} &  $\to$ \tgeoiii{ჳ} \\
23 & \textgeo{უ}  & \tgeoi{უ} & \tgeoii{უ} & u $\to$ \textgeom{u} &  $\to$ \tgeoiii{უ} \\
24 & \textgeo{ფ}  & \tgeoi{ფ} & \tgeoii{ფ} & ph $\to$ \textgeom{ph} &  $\to$ \tgeoiii{ფ} \\
25 & \textgeo{ქ}  & \tgeoi{ქ} & \tgeoii{ქ} & kh $\to$ \textgeom{kh} &  $\to$ \tgeoiii{ქ} \\
26 & \textgeo{ღ}  & \tgeoi{ღ} & \tgeoii{ღ} & gh $\to$ \textgeom{gh} &  $\to$ \tgeoiii{ღ} \\
27 & \textgeo{ყ}  & \tgeoi{ყ} & \tgeoii{ყ} & q $\to$ \textgeom{q} &  $\to$ \tgeoiii{ყ} \\
28 & \textgeo{შ}  & \tgeoi{შ} & \tgeoii{შ} & sh $\to$ \textgeom{sh} &  $\to$ \tgeoiii{შ} \\
29 & \textgeo{ჩ}  & \tgeoi{ჩ} & \tgeoii{ჩ} & tsh $\to$ \textgeom{tsh} &  $\to$ \tgeoiii{ჩ} \\
30 & \textgeo{ც}  & \tgeoi{ც} & \tgeoii{ც} & ts $\to$ \textgeom{ts} &  $\to$ \tgeoiii{ც} \\
31 & \textgeo{ძ}  & \tgeoi{ძ} & \tgeoii{ძ} & j $\to$ \textgeom{j} &  $\to$ \tgeoiii{ძ} \\
32 & \textgeo{წ}  & \tgeoi{წ} & \tgeoii{წ} & c $\to$ \textgeom{c} &  $\to$ \tgeoiii{წ} \\
33 & \textgeo{ჭ}  & \tgeoi{ჭ} & \tgeoii{ჭ} & ch $\to$ \textgeom{ch} &  $\to$ \tgeoiii{ჭ} \\
34 & \textgeo{ხ}  & \tgeoi{ხ} & \tgeoii{ხ} & x $\to$ \textgeom{x} &  $\to$ \tgeoiii{ხ} \\
35 & \textgeo{ჴ}  & \tgeoi{ჴ} & \tgeoii{ჴ} & qh $\to$ \textgeom{qh} &  $\to$ \tgeoiii{ჴ} \\
36 & \textgeo{ჯ}  & \tgeoi{ჯ} & \tgeoii{ჯ} & j $\to$ \textgeom{j} &  $\to$ \tgeoiii{ჯ} \\
37 & \textgeo{ჰ}  & \tgeoi{ჰ} & \tgeoii{ჰ} & h $\to$ \textgeom{h} &  $\to$ \tgeoiii{ჰ} \\
38 & \textgeo{ჵ}  & \tgeoi{ჵ} & \tgeoii{ჵ} & oo $\to$ \textgeom{oo} &  $\to$ \tgeoiii{ჵ} \\
\end{tabular}

\newpage
\uccon
%\geofont 
მზის სისტემა - ვიკიპედია
\uccoff

\geofontipa მზის სისტემა - ვიკიპედია

\geofontiso მზის სისტემა - ვიკიპედია

\geofontpkb მზის სისტემა - ვიკიპედია


%ბჟაშ პლანეტარული სისტემა. ბჟაშ სისტემაშა მიშმურც არძო, მუში მუკი-მუკი მართაფუ რსხული: პლანეტეფი, თინეფიშ ალმაშარეეფი, ასტეროიდეფი დო კომეტეფი. 
%
%    ბჟა
%    პლანეტეფი დო მუში ალმაშარეეფი
%        მერკური
%        მორღა
%        დიხაუჩა
%            თუთა
%        მოროხი
%            ფობოსი დო დეიმოსი
%        დია
%            დიაშ ალმაშარეეფი
%        მარსქული
%            მარსქულიშ ალმაშარეეფი
%            მარსქულიშ რულეეფი
%        ურანი
%            ურანიშ ალმაშარეეფი
%        ნეპტუნი
%            ნეპტუნიშ ალმაშარეეფი
%            
%            
%            
%მზის სისტემა შედგება მზისა და მის გარშემო მოძრავი გრავიტაციულად ჩაჭერილი ასტრონომიული ობიექტებისაგან. მზის სისტემის ფორმირება 4,6 მილიარდი წლის წინ, მოლეკულური ღრუბლის კოლაფსის შედეგად მოხდა. სისტემის მასის უმეტესობას (99,86\%) მზე შეიცავს. ოთხი შედარებით პატარა შიდა პლანეტა — მერკური, ვენერა, დედამიწა და მარსი (მათ ასევე მოიხსენიებენ, როგორც კლდოვანი პლანეტები), ძირითადად, ქვისა და მეტალისგან შედგება, ხოლო ორი უდიდესი პლანეტა — იუპიტერი და სატურნი, ძირითადად, წყალბადითა და ჰელიუმითაა გაჯერებული. ორ უშორეს პლანეტაზე — ურანსა და ნეპტუნზე მეთანის, წყალბადისა და ამიაკის ყინულების დიდი მარაგია, რის გამოც მათ ზოგჯერ „ყინულის გიგანტებად“ მოიხსენიებენ. 
%
%
%%ཉི་མའི་ཁྱིམ་རྒྱུད།
%
%\geofontiso
%ბჟაშ პლანეტარული სისტემა. ბჟაშ სისტემაშა მიშმურც არძო, მუში მუკი-მუკი მართაფუ რსხული: პლანეტეფი, თინეფიშ ალმაშარეეფი, ასტეროიდეფი დო კომეტეფი. 
%
%    ბჟა
%    პლანეტეფი დო მუში ალმაშარეეფი
%        მერკური
%        მორღა
%        დიხაუჩა
%            თუთა
%        მოროხი
%            ფობოსი დო დეიმოსი
%        დია
%            დიაშ ალმაშარეეფი
%        მარსქული
%            მარსქულიშ ალმაშარეეფი
%            მარსქულიშ რულეეფი
%        ურანი
%            ურანიშ ალმაშარეეფი
%        ნეპტუნი
%            ნეპტუნიშ ალმაშარეეფი
%            
%            
%            
%მზის სისტემა შედგება მზისა და მის გარშემო მოძრავი გრავიტაციულად ჩაჭერილი ასტრონომიული ობიექტებისაგან. მზის სისტემის ფორმირება 4,6 მილიარდი წლის წინ, მოლეკულური ღრუბლის კოლაფსის შედეგად მოხდა. სისტემის მასის უმეტესობას (99,86\%) მზე შეიცავს. ოთხი შედარებით პატარა შიდა პლანეტა — მერკური, ვენერა, დედამიწა და მარსი (მათ ასევე მოიხსენიებენ, როგორც კლდოვანი პლანეტები), ძირითადად, ქვისა და მეტალისგან შედგება, ხოლო ორი უდიდესი პლანეტა — იუპიტერი და სატურნი, ძირითადად, წყალბადითა და ჰელიუმითაა გაჯერებული. ორ უშორეს პლანეტაზე — ურანსა და ნეპტუნზე მეთანის, წყალბადისა და ამიაკის ყინულების დიდი მარაგია, რის გამოც მათ ზოგჯერ „ყინულის გიგანტებად“ მოიხსენიებენ. 



\end{document}